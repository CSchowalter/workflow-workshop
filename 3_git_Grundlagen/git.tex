\documentclass{beamer}
\usepackage[utf8]{inputenc}
\usetheme{Rochester}  %% Themenwahl
\usepackage[ngerman]{babel}

% Abschnittseinleitungsfolien einblenden
\AtBeginSection[]{
  \begin{frame}
  \vfill
  \centering
  \begin{beamercolorbox}[sep=8pt,center,shadow=false,rounded=false]{title}
    \usebeamerfont{title}\insertsectionhead\par%
  \end{beamercolorbox}
  \vfill
  \end{frame}
}

\title{SPITZE Workflow 3: Was soll das mit dem git?}
\author{Christian Schowalter}
\date{\today}

\begin{document}
\maketitle
%\frame{\tableofcontents[currentsection]}

\section{Einführung}

\begin{frame}
	\frametitle{Vorwort}
	\begin{block}{Warum git?}
		\begin{itemize}
			\item dezentral
			\item flexibel
			\item simpel
			\item schnell
			\item unterstützt nicht-lineares Arbeiten
		\end{itemize}
	\end{block}
\end{frame}

\begin{frame}
	\frametitle{Grundlagen}
	\begin{block}{ein paar Begriffe vorweg}
		\begin{itemize}
			\item Repository
			\item Commit
			\item Diff
		\end{itemize}
	\end{block}
\end{frame}

\begin{frame}
	\frametitle{Einschränkungen}
	\begin{block}{was nicht (gut) geht}
		\begin{itemize}
			\item Binärdateien verwalten
			\item Deployment auf Produktivserver
			\item Passwörter vergessen
		\end{itemize}
	\end{block}
\end{frame}

\begin{frame}
	\includegraphics[width=\textwidth]{images/dont_panic.jpg}
\end{frame}

\begin{frame}
Was git alleine nicht schafft, schafft git mit anderen Tools.
\end{frame}

\section{lokales Handwerkszeug}

\begin{frame}
	\frametitle{Repositorien}
	\begin{block}{Was ist überhaupt ein Repository?}
		\pause
		lokaler Ordner, der von git verwaltet wird
	\end{block}
	\pause
	\begin{block}{… und was ist da so drin?}
		\begin{itemize}
			\item Konfigurationsdateien
				\pause
			\item das was ihr da reintut
		\end{itemize}
	\end{block}
\end{frame}

\begin{frame}
	\frametitle{Der Commit, das unbekannte Wesen}
	\begin{block}{Woraus besteht ein Commit?}
		\begin{itemize}
			\item Commit Message
				\pause
			\item Snapshot der Dateiinhalte
				\pause
			\item Prüfsumme
				\pause
			\item Metadaten (Autor, Mail-Adresse, Datum)
		\end{itemize}
	\end{block}
	\pause
	\begin{block}{… und wozu soll das gut sein?}
		\begin{itemize}
			\item Grundbaustein der Versionierung
				\pause
			\item Entwicklungsgeschichte
				\pause
			\item Vertrauensbasis
		\end{itemize}
	\end{block}
\end{frame}

\begin{frame}
	\frametitle{Der git Commit in seiner natürlichen Umgebug}
	\begin{block}{Was man mit diesen Commits alles tun kann}
		\begin{itemize}
			\item log anschauen
				\pause
			\item gezielt auschecken
				\pause
			\item untereinander per Diff vergleichen
				\pause
			\item branchen
				\pause
			\item einiges weiteres (z.T.\ advanced)…
		\end{itemize}
	\end{block}
\end{frame}

\begin{frame}
	\frametitle{Parallelwelten}
	\begin{block}{Branch}
		\begin{itemize}
			\item basieren auf jew. einen Commit
				\pause
			\item Branches können…
				\begin{itemize}
					\item gebrancht werden
					\item zusammengeführt werden
					\item nachträglich auf anderen Commit basieren
					\item verglichen werden
				\end{itemize}
				\pause
			\item Wechsel zwischen Branches quasi jederzeit möglich
		\end{itemize}
	\end{block}
\end{frame}

\section{Remote Repositories}

\begin{frame}
%	\includegraphics[width=\textwidth]{images/WelcometotheInternet.jpg} %TODO: Insert your favorite funpic here
\end{frame}

\begin{frame}
	\frametitle{Repositorien von einem anderen Stern}
	\begin{block}{remote repositories}
		\begin{itemize}
			\item haben die gleichen Features wie das lokale
			\item flexibel, nachträglich änderbar
			\item eigener Server oder GitLab, GitHub, Bitbucket…
		\end{itemize}
	\end{block}
\end{frame}

\begin{frame}
	\begin{center}
		Und dann ist automatisch alles online?
		\pause
%		\includegraphics[width=\textwidth]{images/hahaha_no.jpg} %TODO: Add another fine pic for your audience to enjoy
	\end{center}
\end{frame}

\begin{frame}
	\frametitle{Mit dem Remote auf Du und Du}
	\begin{block}{Wie arbeitet man online?}
		\begin{itemize}
			\item Synchronisiation mit remote repo passiert manuell
				\pause
				\begin{itemize}
					\item es gibt (Windows-) Tools, die das automatisieren. Finger Weg davon!
					\item Phasenweise ohne Internetzugang arbeiten problemlos
				\end{itemize}
				\pause
			\item Pushen und Pullen als getrennte Aktionen
		\end{itemize}
	\end{block}
\end{frame}

\begin{frame}
	\frametitle{alles public, oder was?}
	\begin{block}{Ein git repo als SPOT}
		\begin{itemize}
			\item gemeinsames remote repository für das ganze Team
				\pause
			\item Regeln notwendig
				\pause
				\begin{itemize}
					\item Es gibt Rechteverwaltung in GitLab u.ä.
					\item Arbeitsprozesse müssen trotzdem definiert ablaufen
				\end{itemize}
				\pause
			\item Dazu mehr in einem eigenen Workshopteil (oder so…)
		\end{itemize}
	\end{block}
\end{frame}

\begin{frame}
	\frametitle{Webkram}
	\begin{block}{GitLab, GitHub, Bitbucket undsoweiter}
		\begin{itemize}
			\item bieten grundsätzlich ein normales git remote repository an
			\item diverse Zusatzfeatures (z.T. nur im Browser)
			\item unterschiedliche Nutzungsbedingungen
				\begin{itemize}
					\item z.T. kostenpflichtig
				\end{itemize}
		\end{itemize}
	\end{block}
\end{frame}

\section{Fazit}

\begin{frame}
	\frametitle{Next up: Hands on}
	\begin{center}
		Wie geht es weiter?
	\end{center}
\end{frame}

%git branch

%git remote

\end{document}
