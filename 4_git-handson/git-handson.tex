\documentclass{beamer}
\usepackage[utf8]{inputenc}
\usetheme{Rochester}  %% Themenwahl
\usepackage[ngerman]{babel}

% Abschnittseinleitungsfolien einblenden
\AtBeginSection[]{
  \begin{frame}
  \vfill
  \centering
  \begin{beamercolorbox}[sep=8pt,center,shadow=false,rounded=false]{title}
    \usebeamerfont{title}\insertsectionhead\par%
  \end{beamercolorbox}
  \vfill
  \end{frame}
}

\title{SPITZE Workflow 4: git gud!}
\author{Christian Schowalter}
\date{\today}

\begin{document}
\maketitle
%\frame{\tableofcontents[currentsection]}

\section{Vorbereitung}

\begin{frame}
	\frametitle{Themen}
	\begin{block}{alltägliches}
		\begin{itemize}
			\item Dateien verschieben
			\item Fehler beheben 
			\item Hotfixen
			\item Taggen
		\end{itemize}
	\end{block}
\end{frame}

\begin{frame}
	\frametitle{Spielregeln}
	\begin{block}{Schummeln erlaubt}
		\begin{itemize}
			\item git Spickzettel 
			\item Google fragen
			\item gegenseitig helfen
		\end{itemize}
	\end{block}
	\pause
	Der Clou: Ihr müsst die Befehle selbst herausfinden.
\end{frame}


\begin{frame}
	\frametitle{Warm-Up}
	\begin{block}{geht direkt los hier}
		\begin{itemize}
			\item git pull
			\item git checkout -b ???
			\item git push ???
		\end{itemize}
	\end{block}
\end{frame}

\section{private Branch}


\begin{frame}
	\frametitle{Aufgabe 1}
	\begin{block}{Move Commit}
		Die Datei \texttt{foo.txt} soll umbenannt werden in \texttt{bar.txt}.
		Benennt diese um und macht aus dieser Änderung einen Commit!
		Verwendet dabei ausschlielich git Befehle!
	\end{block}
\end{frame}

\begin{frame}
	\frametitle{Aufgabe 2}
	\begin{block}{Upps!}
		\begin{enumerate}
			\item Löscht den Inhalt von Datei \texttt{gedicht.txt} und schreibt Buchstabensalat!
			\pause
			\item Abspeichern! (im Texteditor, noch nix mit git)
			\pause
			\item Oh nein, lauter Buchstabensalat… Stellt den vorherigen Stand der Datei mit git wieder her!
		\end{enumerate}
	\end{block}
\end{frame}

\begin{frame}
	\frametitle{Aufgabe 3}
	\begin{block}{Großes Upps!}
		\begin{enumerate}
			\item Datei \texttt{humor.txt} muss dieses Mal dran glauben. Buchstabensalat!
			\pause
			\item Abspeichern!
			\pause
			\item \texttt{git add}, \texttt{git commit}!
			\pause
			\item Diese Änderung war natürlich ungünstig.
				Stattdessen soll in der Datei der Kurze Text „\texttt{Hallo, Welt!}“ stehen.
				Korrigiert den Commit, ohne einen neuen zu machen!
		\end{enumerate}
	\end{block}
\end{frame}

\begin{frame}
	\frametitle{Aufgabe 4}
	\begin{block}{Weg damit!}
		Die Datei \texttt{useless.txt} gefällt uns auf einmal gar nicht mehr.
		Löscht sie mit git und macht daraus einen Commit!
	\end{block}
\end{frame}

\section{public Branch}

\begin{frame}
	\frametitle{Gemeinsam genutzte Branches}
	\begin{block}{Achtung, wir schreiben Geschichte.}
		\begin{itemize}
			\item Commits dürfen nicht verändert oder entfernt werden
				\begin{itemize}
					\item git-commit ohne --amend
					\item kein git-rebase
					\item git-push --force ist Tabu
					\item Rückgängig-machen grundsätzlich per neuem Commit
				\end{itemize}
			\item Reihenfolge der Commits bewahren!
				\begin{itemize}
					\item public Branches nicht in andere Branches mergen
					\item stattdessen rebase im private
				\end{itemize}
		\end{itemize}
	\end{block}
\end{frame}

\begin{frame}
	\frametitle{Aufgabe 1}
	\begin{block}{Wer macht denn sowas?}
		Im aktuellen Commit steht in der Datei \texttt{repariermich.txt} nur Dummes Zeug™.
		Der Commit davor ist viel besser.
		Sein Zustand soll wiederhergestellt werden.
		Repariert es bei euch so, dass ihr es reinen Gewissens pushen könnt.
	\end{block}
\end{frame}

\begin{frame}
	\frametitle{Aufgabe 2}
	\begin{block}{Fix it like it’s hot!}
		Die Datei \texttt{hotfixbitte.txt} ist in den Branches master und preview enthalten (beide public).
		Das Wort \texttt{foo} in Zeile 3 soll in beiden Branches auf \texttt{bar} geändert werden.
		Nutzt dazu einen Hotfix-Branch!
	\end{block}
\end{frame}

\begin{frame}
	\frametitle{Aufgabe 3}
	\begin{block}{Stabil}
		Der aktuelle Zustand des master-Branches erfüllt die Qualitätskriterien.
		Die Versionsnummer soll jetzt 1.0 lauten und die Message „\texttt{My work here is done.}“ enthalten.
	\end{block}
\end{frame}
\end{document}
