\documentclass{beamer}
\usepackage[utf8]{inputenc}
\usetheme{Warsaw}  %% Themenwahl
\usepackage[ngerman]{babel}

\title{SPITZE Workflow}
\author{Christian Schowalter}
\date{\today}

\begin{document}
\maketitle
%\frame{\tableofcontents[currentsection]}

\section{Konzepte}
\begin{frame} %%Eine Folie
  \frametitle{Vorwort}
  \begin{Definition}
  	\begin{itemize}
  	\item Dieser Teil des Vortrags soll möglichst ohne die Nennung konkreter Tools ablaufen.
    \item Kann offensichtlich wirken. Das ist ein gutes Zeichen.
    \item Überlegt euch bitte, ob und wie die Konzepte in eurem Fall anwendbar sind!
    \end{itemize}
  \end{Definition}
\end{frame}

\begin{frame}
	\begin{block}{DRY}
		\begin{itemize}
			\item Don’t repeat yourself!
			\item Gegenteil von WET (\glqq write everything twice\grqq , \glqq we enjoy typing\grqq , \glqq waste everyone’s time\grqq)
			\item Wenn Änderungen an einer Stelle nötig sind, sollte das keine weiteren Änderungen bedingen
		\end{itemize}
	\end{block}
\end{frame}

\begin{frame}
	\begin{block}{SPOT}
		\begin{itemize}
			\item Es gibt genau einen Ort, an dem der aktuelle Stand eines bestimmten Arbeitsergebnisses zu finden ist
			\item Betrifft nicht Work in Progress
		\end{itemize}
	\end{block}
\end{frame}

\begin{frame}
	\begin{block}{Staging}
		\begin{itemize}
			\item Stufen auf dem Weg zur offiziellen Veröffentlichung
			\item Funktioniert nur sauber wenn man einen SPOT hat
			\item Bedeutung der Stages ist voher(!) zu definieren.
			\item klare Bedingungen, nicht irgendwie nach Gefühl
			\item üblich:
			\begin{enumerate}
				\item Development
				\item Testing
				\item Production
			\end{enumerate}


		\end{itemize}
	\end{block}
\end{frame}

\begin{frame}
	\begin{block}{Dezentralisierung}
		\begin{itemize}
			\item Man kann den aktuellen Zwischenstand nehmen und für sich weiterarbeiten
			\item Man sollte sich dabei nicht blockieren
			\item Es muss einen Weg geben, die Arbeitsergebnisse zusammenzuführen (mergen) oder zu synchronisieren
			\item Berücksichtigen, dass es einen SPOT gibt, wo man die Ergebnisse hinführt
		\end{itemize}
	\end{block}
\end{frame}

\begin{frame}
	\begin{block}{Versionierung}
		\begin{itemize}
			\item Zwischenergebnisse werden so verwaltet, dass man reverten kann
			\item mit Kommentar wird Nachvollziehbarkeit erreicht
			\item Versionen können Stages zugeordnet werden
		\end{itemize}
	\end{block}
\end{frame}

\begin{frame}
	\begin{block}{Interoperabilität}
		\begin{itemize}
			\item Selten deckt ein Tool alles ab, was man tun können will, also notwendig
			\item Einschränkungen rechtzeitig feststellen!
			\item Betrifft u.a. Dateitypen
			\begin{itemize}
				\item offene Formate
			\end{itemize}
		\end{itemize}
	\end{block}
\end{frame}

\section{Dokumentation}

\begin{frame}
	\begin{block}{Doku… wann, wo, wie?}
		\begin{itemize}
			\pause
			\item jetzt
			\pause
			\item überall
			\pause
			\item ordentlich

		\end{itemize}
	\end{block}
\end{frame}

\begin{frame}
	\begin{block}{Dokumentation, ernsthaft jetzt}
		\begin{itemize}
			\pause
			\item möglichst früh anfangen. Nachtragen ist Krieg!
			\item geschieht nicht nur in dem eigentlichen Dokument
			\begin{description}
				\item[Beispiel:] nachvollziehbare Versionen dokumentieren Erfahrungswerte und Arbeitsweise
			\end{description}
			\item sollte in den Workflow integriert sein
			\item möglichst nicht lästig
		\end{itemize}
	\end{block}
\end{frame}

\begin{frame}
	\begin{block}{Metadokumentation}
		\begin{itemize}
		\item viel zu mühsam, zu dokumentieren, wie man dokumentiert
		\item \href{https://opensource.com/open-organization/17/10/readme-maturity-model?sc_cid=70160000001273HAAQ}{irgendwas im Web anklicken und alles wird gut}
		\end{itemize}
	\end{block}
\end{frame}

\end{document}
